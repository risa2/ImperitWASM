\documentclass[a4paper,12pt]{article}
\usepackage[margin=2cm]{geometry}
\usepackage[czech]{babel}
\usepackage[utf8]{inputenc}
\usepackage{amsmath}
\usepackage{amssymb}
\usepackage{graphicx}
\usepackage{fancyhdr}
\usepackage{lipsum}
\usepackage[backend=bibtex,sorting=none]{biblatex}
\usepackage{tikz}

\usetikzlibrary{shapes.geometric, arrows}
\tikzstyle{block} = [rectangle, minimum width=3cm, minimum height=1cm,text centered, draw=black, fill=red!20]
\tikzstyle{arrow} = [thick,->,>=stealth]
\addbibresource{ZMP.bib}

\def\max #1{\textrm{max}\left\{#1\right\}}
\makeatletter
\newcount\my@repeat@count
\newcommand{\repeatchar}[2]{%
  \begingroup
  \my@repeat@count=\z@
  \@whilenum\my@repeat@count<#1\do{#2\advance\my@repeat@count\@ne}%
  \endgroup
}
\makeatother

\author{Richard Blažek}
\setlength{\headheight}{15pt}
\pagestyle{fancy}
\fancyhead{}
\fancyhead[R]{Imperit}
\fancyhead[L]{Richard Blažek}
\fancyfoot{}
\fancyfoot[R]{\thepage}
\fancyfoot[L]{Sekce \thesection}
\setlength{\parindent}{0pt}
\setlength{\parskip}{0.8em}

\begin{document}
\begin{titlepage}
    \begin{center}

	\vspace*{3cm}            
	\Huge
	\textbf{Imperit}
            
	\vspace{0.5cm}
	\LARGE
	Strategická tahová hra
        
	\vspace*{1cm}
	\Huge
	\textbf{Imperit}
            
	\vspace{0.5cm}
	\LARGE
	Turn-based strategy
            
	\vfill
            
	\large
        Závěrečná maturitní práce, rok 2021\\
	Richard Blažek\\
	Gymnázium Brno, třída Kapitána Jaroše 14
    \end{center}
\end{titlepage}
\thispagestyle{empty}
\Large\textbf{Prohlášení}\normalsize

Prohlašuji, že jsem svou závěrečnou maturitní práci vypracoval samostatně a použil jsem pouze prameny a literaturu uvedené v seznamu bibliografických záznamů.

Prohlašuji, že tištěná verze a elektronická verze závěrečné maturitní práce jsou shodné.

Nemám závažný důvod proti zpřístupňování této práce v souladu se zákonem č. 121/2000 Sb., o právu autorském, o právech souvisejících s právem autorským a o změně některých zákonů (autorský zákon) ve znění pozdějších předpisů. 

V Brně dne \today{} \repeatchar{40}{.}
\newpage
\thispagestyle{empty}
\Large\textbf{Poděkování}\normalsize

Tímto bych chtěl poděkovat Mgr. Marku Blahovi za odborné vedení práce.
\newpage
\thispagestyle{empty}
\Large\textbf{Anotace}\normalsize

Práce se zabývá vytvořením internetové počítačové hry zvané Imperit pomocí Blazor WebAssembly z frameworku .NET. Hra je koncipovaná jako tahová strategie, jejímž tématem je dobývání území na herním plánu.

\Large\textbf{Klíčová slova}\normalsize

počítačová hra;tahová strategie;blazor;webassembly;dotnet

\Large\textbf{Annotation}\normalsize

The thesis is concerned about creation of online browser-based game called Imperit using Blazor WebAssembly from the .NET framework. The game is designed as a turn-based strategy consisting of conquering the territory on the game map.

\Large\textbf{Keywords}\normalsize

computer game;turn-based strategy;blazor;webassembly;dotnet
\newpage
\thispagestyle{empty}
\tableofcontents
\newpage
\section{Úvod}
Rozhodl jsem se vytvořit hru o dobývání území navrženou tak, aby byla konečná a vítěz byl jednoznačně určen. Vítězství by však nemělo na konci hry nastat neočekávaně, již v průběhu hry bude patrný vývoj, z něhož vyplyne, který hráč má k výhře nejblíže, ale tento vývoj budou moci ostatní hráči zvrátit. Hra by také měla obsahovat náhodný prvek, ovšem pravděpodobnost náhodných jevů by měla být známá, aby byli hráči nuceni s rizikem počítat.

Pro vytvoření hry jsem zvolil formát tahové strategie, neboť strategie v reálném čase buď vyvíjí tlak na hráče z důvodu nedostatku času, nebo ve snaze vyhnout se tomuto problému vede ke zdlouhavému čekání na dokončení některých akcí \cite{turnreal1}. V obou případech realtimová strategie vede k orientaci na postřeh a trpělivost spíše než na vymýšlení strategie\cite{turnreal2}. Kromě toho \uv{připoutává} hráče k jejich obrazovkám, což sice může být mnohdy žádoucí, ale můj záměr takový nebyl.

\section{Popis hry}
\subsection{Provincie}
Hra se odehrává na plánu, který se skládá z provincií (země, moře a pohoří), jejichž názvy a tvary jsou zvolené podle evropských zemí a moří nebo jejich části. Provincie obsahují vojenské jednotky a jsou ovládány nejvýše jedním hráčem, který v nich může verbovat jednotky. Země se mohou odtrhnout a pravděpodobnost, že se tak v daném kole stane je $P=\max{\displaystyle\frac{P_0\cdot (S_0-S)}{S_0},0}$, kde $P_0$ je výchozí pravděpodobnost odtržení (příbližně 10 \%), $S$ je obranná síla jednotek přítomných v zemi a $S_0$ je obranná síla výchozích jednotek, jež byly v zemi na začátku hry. Provincie je dobyta, je-li napadena vojskem, jehož útočná síla je vyšší, než je obranná síla jednotek v provincii. Přístavy jsou druhem zemí, které umožňují přesun na moře a z moře.
\subsection{Hráči}
Hráčů může být $2$-$16$. Při registraci si zvolí zemi, v níž budou začínat, a na začátku hry dostanou určité množství peněz. Dále získají v každém kole určité množství peněz za každou zemi, kterou ovládají, přičemž toto množství se u různých zemí liší. Za peníze si mohou kupovat země a verbovat ve svých provinciích vojenské jednotky.
\subsection{Vojenské jednotky}
Vojenské jednotky slouží k dobývání provincií. Každá jednotka má určitou cenu, sílu v obraně, sílu v útoku a hmotnost, jež je relevantní v případě, že jednotka nemůže provést určitý přesun samotná a potřebuje například loď k přesunu přes moře. Jednotky schopné přenášet jiné (např. loď, slon) mají též nosnost, jež udává, jaká může být celková hmotnost jednotek, které se s ní přepravují.
\subsection{Průběh hry}
Hra probíhá po kolech, během nichž mají všichni hráči postupně svůj tah. Hráč není v rámci svého tahu neomezen časem ani počtem akcí, ale verbování a přesuny vojska se provedou až po ukončení tahu.
\subsection{Cíl hry}
Cílem hry je ovládnout čtyři z pěti cílových provincií, které jsou viditelně označené a jsou rozprostřeny po všech okrajích mapy, takže k dobytí všech je nutné ovládnout podstatnou část mapy, ačkoli hráč může při vhodné strategii vyhrát, aniž by měl největší území.

\section{Použité technologie}
\subsection{C\#}
Celý program je napsaný v jazyce C\#, neboť tento jazyk může běžet na straně serveru (viz ASP.NET Core) i klienta (viz Blazor WebAssembly) a obě části mohou využívat společné knihovny a pracovat s týmiž datovými typy. Navíc tento jazyk umožňuje využívat knihoven z frameworku .NET a na rozdíl od jazyků PHP a JavaScript, které se často pro vývoj webových aplikací používají, je staticky typovaný a kompilovaný do bytekódu, což do určité míry kontroluje správnost programu.
\subsection{ASP.NET Core}
Framework ASP.NET Core je využíván na straně serveru ke zpracovávání HTTP dotazů a odesílání odpovědí.
\subsection{Blazor WebAssembly}
Blazor WebAssembly je framework, který umožňuje vyvíjet v jazyce C\# aplikace spustitelné v prohlížeči. Kód v jazyce C\# se přeloží do binárního formátu WebAssembly a prohlížeč pomocí krátkého kódu v JavaScriptu výsledný soubor stáhne a spustí. Spuštěný program následně reaguje na akce uživatele, komunikuje se serverem pomocí HTTP a podle potřeby překresluje zobrazovanou webovou stránku.

\section{Struktura}
Program se skládá ze tří projektů: Server, Client a Shared. Projekt Shared obsahuje třídy, které druhé dva projekty využívají, a nachází se v něm většina herní logiky. Projekt Server obsahuje kód spouštěný na straně serveru. Projekt Client obsahuje kód překládaný do WebAssembly a spouštěný v prohlížeči klienta.

\begin{tikzpicture}[node distance=2cm]
\node (shared) [block] {Shared};
\node (client) [block, right of=shared, xshift=2cm] {Client};
\node (server) [block, below of=shared] {Server};
\draw [arrow] (client) -- (shared);
\draw [arrow] (server) -- (shared);
\draw [arrow] (server) -- (client);
\end{tikzpicture}
\subsection{Shared}

\newpage
\printbibliography[heading=bibintoc, title={Použitá literatura}]
\end{document}